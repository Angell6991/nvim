\documentclass[10pt,a4paper]{article}
\usepackage[utf8]{inputenc}
\usepackage[spanish]{babel}
\usepackage{amsmath}
\usepackage{amsfonts}
\usepackage{amssymb}
\usepackage{graphicx}
\usepackage{txfonts}
\usepackage{mathrsfs}
\usepackage{color}
\usepackage{parskip}
\usepackage[spanish]{babel}
\usepackage[left=2cm,right=2cm,top=2cm,bottom=2cm]{geometry}
\author{Nombre del Autor}

%definiendo notación para los braket´s y spin´s
\newcommand{\spin}[1]{\boldsymbol{#1}}
\newcommand{\bra}[1]{\left\langle {#1} \right|}
\newcommand{\ket}[1]{\left| {#1} \right\rangle}
\newcommand{\esperado}[1]{\left \langle  {#1} \right\rangle}
\newcommand{\braket}[2]{\left \langle  {#1} \mid {#2} \right\rangle}

%definiendo coorchete de Poisson, conmutador y anticonmutador
\newcommand{\poisson}[2]{ \left\lbrace  {#1},{#2} \right\rbrace  }
\newcommand{\conm}[2]{\left[ {#1},{#2} \right]}
\newcommand{\antconm}[2]{\left(\; {#1}, {#2} \; \right)}


%definiendo la norma y el determinante
\newcommand{\norma}[1]{\left| \; {#1} \; \right|}
\newcommand{\Det}[1]{\left| \left| \; {#1} \; \right|\right|}

%Definiendo la forma del Jacobiano
\newcommand{\jac}[2]{\mathcal{J}_{\;\;#1}^{#2}}
\newcommand{\jacinverse}[2]{\mathcal{J}_{#1}^{\;\;#2}}

%Definiendo parte real y imaginaria%
\newcommand{\reaal}[1]{Re\left\lbrace \; {#1} \; \right\rbrace}
\newcommand{\imag}[1]{Im\left\lbrace \; {#1} \; \right\rbrace}
	

\begin{document}

\begin{center}
UNIVERSIDAD DISTRITAL FRANCISCO JOSE DE CALDAS
\end{center}


\begin{center}
LICENCIATURA EN FISICA
\end{center}


\begin{center}
Nombre del Autor
\end{center}


\begin{center}
Numero de identificacion
\end{center}



\section{Nombre de la seccion}  	
% \input{~/Descargas/equation.tex} %input es para agregar el contenido de otro documento del tipo .tex ,"precaución con la ruta del archivo"
\label{sec1} %label es para etiqueter

% \ref{sec1} %ref llamamos la etiqueta


% Ecuacion:
% \begin{equation}
% * \; * \; *
% \label{eq1}
% \end{equation}


% Matriz:
% \left( \;
% \begin{matrix}
% 	* & * \\
% 	* & * \\
% 	* & * 
% \end{matrix}
% \; \right)


% Tabla:
% \begin{tabular}{|c|c|}
%	\hline
%	* & * \\
%	\hline
%	* & * \\
%	\hline
% \end{tabular}

% Enumerar:
% \begin{enumerate}
% \item
% \item
% \item
% \end{enumerate}


% Imagen:
% \begin{figure}[h]
%	\includegraphics[width=0.7\linewidth, height=0.7\textheight]{~/Imágenes/Imagenes/00wallpaper03.jpg}
	%\includegraphics[scale=1]{~/Imágenes/Imagenes/00wallpaper03.jpg}
%	\caption{descripcción}
	%\label{imag1}
% \end{figure}


% Bibliografia:
% \begin{thebibliography}{00}
% \bibitem{REF_1} 
% \bibitem{REF_2} 
% \bibitem{REF_4}
% \bibitem{REF_5}
% \bibitem{REF_6}
% \end{thebibliography}



\end{document}
