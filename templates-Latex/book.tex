\documentclass[10pt,a4paper,openany]{book}
\usepackage[utf8]{inputenc}
\usepackage[spanish]{babel}
\usepackage{amsmath}
\usepackage{amsfonts}
\usepackage{amssymb}
\usepackage{graphicx}
\usepackage{txfonts}
\usepackage{mathrsfs}
\usepackage{color}
\usepackage{setspace}
\usepackage{cancel}
\usepackage{parskip}
\usepackage[left=2.54cm,right=2.54cm,top=2.54cm,bottom=2.54cm]{geometry}
\author{Miguel Angel Rodríguez Guzmán}


%definiendo notación para los braket´s y spin´s
\newcommand{\spin}[1]{\boldsymbol{#1}}
\newcommand{\bra}[1]{\left\langle {#1} \right|}
\newcommand{\ket}[1]{\left| {#1} \right\rangle}
\newcommand{\esperado}[1]{\left \langle  {#1} \right\rangle}
\newcommand{\braket}[2]{\left \langle  {#1} \mid {#2} \right\rangle}

%definiendo coorchete de Poisson, conmutador y anticonmutador
\newcommand{\poisson}[2]{ \left\lbrace  {#1},{#2} \right\rbrace  }
\newcommand{\conm}[2]{\left[ {#1},{#2} \right]}
\newcommand{\antconm}[2]{\left(\; {#1}, {#2} \; \right)}


%definiendo la norma y el determinante
\newcommand{\norma}[1]{\left| \; {#1} \; \right|}
\newcommand{\Det}[1]{\left| \left| \; {#1} \; \right|\right|}

%Definiendo la forma del Jacobiano
\newcommand{\jac}[2]{\mathcal{J}_{\;\;#1}^{#2}}
\newcommand{\jacinverse}[2]{\mathcal{J}_{#1}^{\;\;#2}}

%Definiendo parte real y imaginaria%
\newcommand{\reaal}[1]{Re\left\lbrace \; {#1} \; \right\rbrace}
\newcommand{\imag}[1]{Im\left\lbrace \; {#1} \; \right\rbrace}
	


\begin{document}

%\begin{titlepage}
\begin{center}
{\LARGE \textbf{UNIVERSIDAD DISTRITAL FRANCISCO JOSE DE CALDAS}}\\

\vspace{2mm}

{\large \textbf{UBI VERITAS IBI LIBERTAS}}\\

%insertar imagen para la Portada
%\begin{figure}[h]
%	\centering
%	\includegraphics[scale=0.3]{Portada}
%\end{figure}

\rule{150mm}{0.5mm}
\vspace{2mm}
\begin{spacing}{2}
	{\Large Dinámica Gravitacional de una Partícula Cuántica }
\end{spacing}
\vspace{2mm}
\rule{150mm}{0.5mm}


\vspace{2cm}
{\Large \textbf{Nombre del Autor}}\\
\vspace{5mm}
{\large Facultad de Ciencias y Educación}\\
\vspace{5mm}
{\Large Licenciatura en Física}\\
\vspace{5mm}
{\Large Colombia}\\
\vspace{5mm}
{\Large Bogotá D.C}

\vfil
{\Huge \textbf{2022}}

\end{center}
\end{titlepage}
 %incluir Portada Nota: el archivo de la portada debe de estar e la misma carpeta dende se aloje este archivo

\tableofcontents
	
\chapter*{Introducción}
	%\input{Intro}
	\addcontentsline{toc}{section}{ \textbf{Introducción}}
	
		
\chapter{Nombre del capitulo}
	%\input{1 Cap Relatividad general/Relatividad General}
	
	\section{Nombre de la seccion}  
	%\label{sec1} %etiqueta
	%\input{1 Cap Relatividad general/Vectores y C Vectores}
	
	\subsection{Nombre de la subseccion}
	%\input{1 Cap Relatividad general/C Posiscion}
	
	
\end{document}
